\documentclass[]{article}
\usepackage{lmodern}
\usepackage{amssymb,amsmath}
\usepackage{ifxetex,ifluatex}
\usepackage{fixltx2e} % provides \textsubscript
\ifnum 0\ifxetex 1\fi\ifluatex 1\fi=0 % if pdftex
  \usepackage[T1]{fontenc}
  \usepackage[utf8]{inputenc}
\else % if luatex or xelatex
  \ifxetex
    \usepackage{mathspec}
  \else
    \usepackage{fontspec}
  \fi
  \defaultfontfeatures{Ligatures=TeX,Scale=MatchLowercase}
\fi
% use upquote if available, for straight quotes in verbatim environments
\IfFileExists{upquote.sty}{\usepackage{upquote}}{}
% use microtype if available
\IfFileExists{microtype.sty}{%
\usepackage{microtype}
\UseMicrotypeSet[protrusion]{basicmath} % disable protrusion for tt fonts
}{}
\usepackage[unicode=true]{hyperref}
\hypersetup{
            pdfborder={0 0 0},
            breaklinks=true}
\urlstyle{same}  % don't use monospace font for urls
\usepackage{longtable,booktabs}
\usepackage{graphicx,grffile}
\makeatletter
\def\maxwidth{\ifdim\Gin@nat@width>\linewidth\linewidth\else\Gin@nat@width\fi}
\def\maxheight{\ifdim\Gin@nat@height>\textheight\textheight\else\Gin@nat@height\fi}
\makeatother
% Scale images if necessary, so that they will not overflow the page
% margins by default, and it is still possible to overwrite the defaults
% using explicit options in \includegraphics[width, height, ...]{}
\setkeys{Gin}{width=\maxwidth,height=\maxheight,keepaspectratio}
\IfFileExists{parskip.sty}{%
\usepackage{parskip}
}{% else
\setlength{\parindent}{0pt}
\setlength{\parskip}{6pt plus 2pt minus 1pt}
}
\setlength{\emergencystretch}{3em}  % prevent overfull lines
\providecommand{\tightlist}{%
  \setlength{\itemsep}{0pt}\setlength{\parskip}{0pt}}
\setcounter{secnumdepth}{0}
% Redefines (sub)paragraphs to behave more like sections
\ifx\paragraph\undefined\else
\let\oldparagraph\paragraph
\renewcommand{\paragraph}[1]{\oldparagraph{#1}\mbox{}}
\fi
\ifx\subparagraph\undefined\else
\let\oldsubparagraph\subparagraph
\renewcommand{\subparagraph}[1]{\oldsubparagraph{#1}\mbox{}}
\fi

\date{}

\begin{document}

\textbf{Instituto Federal de Santa Catarina - IFSC}

\textbf{Engenharia de Controle e Automação}

\textbf{Projeto Integrador}

\protect\hypertarget{_rcmtlgrobc4b}{}{}Mesa Birrotativa Para Solda Com
Parâmetros Controlados

\textbf{Aécio Alves Da Anunciação}

\textbf{Guilherme Simioni Tauffer}

\textbf{Eduardo Cinésio Costa da Silva }

\textbf{Chapecó, 2022}

\protect\hypertarget{_2k5acx2zp81h}{}{}

\section{Introdução }\label{introduuxe7uxe3o}

Com a rápida evolução tecnológica, e com a alta competitividade,
desenvolver um produto e introduzi-lo no mercado está tornando-se cada
vez mais desafiador.

Não basta apenas inovar, é necessário manter a competitividade e criar
um produto que se solidifique no mercado, conquistando clientes e
apreço.

Segundo Dym, ``Um bom projeto não acontece por acaso. Em vez disso, ele
resulta do pensamento cuidadoso sobre o que os clientes e usuários
querem e sobre como anunciar e alcançar os requisitos do projeto.''.

Deste modo, não basta apenas conhecer o problema, mas também é
necessário conhecer as mais variadas soluções possíveis, buscando
informações e integrando ao projeto profissionais de diferentes áreas,
os quais devem corroborar aplicando o conhecimento para cada área
envolvida.

Para que um projeto tenha sucesso é necessário um bom gerenciamento do
projeto, assim, obtém-se bons resultados.

Desta forma, este trabalho contém o projeto informacional realizado com
base na metodologia de projeto e produto adotado pelo Instituto Federal
de Santa Catarina, de uma mesa birrotativa com parâmetros controlados,
que terá como cliente final a própria instituição de ensino. O intuito é
utilizar da criação da máquina desenvolvida neste projeto para aplicação
de soldas de maneira controlada e automatizada.

\section{Projeto Preliminar}\label{projeto-preliminar}

Uma mesa birrotativa de soldagem é uma máquina que facilita a realização
de tarefas de solda em peças tubulares ou de geometrias mais complexas.
Esta máquina foi criada com o intuito de atender a necessidade das
indústrias, aumentando a eficiência de produção.

Atualmente, na indústria 4.0 houve a automatização das linhas
produtivas, substituindo a mão de obra humana por robôs e braços
robóticos, os quais trabalham em conjunto e de forma muito eficiente com
a mesa birrotativa, rotacionado a peça a ser soldada junto ao movimento
do robô. Esse tipo de tecnologia é muito implementado nas indústrias
metalúrgicas, automotivas, aeroespacial e naval.

Para a realização do projeto da mesa birrotativa, primeiramente, é
necessário levantar as definições do problema e as demandas do projeto,
como é o que a mesa deve cumprir e suprir no final de sua construção.
Tendo como pré requisitos de projeto dois graus de liberdade
controlados, ou seja controle de velocidade e posição no \textbf{Eixo 1}
e controle de posição no \textbf{Eixo 2}, e a integração com o robô ABB
IRB140. Visando isso, o protótipo mostrado abaixo foi desenvolvido.

\paragraph{\texorpdfstring{Figura 1. Representação dos eixos na mesa
projetada. (Autoria
própria).\protect\includegraphics[width=1.96354in,height=1.10622in]{media/image34.png}\protect\includegraphics[width=2.67295in,height=3.40037in]{media/image30.png}}{Figura 1. Representação dos eixos na mesa projetada. (Autoria própria).}}\label{figura-1.-representauxe7uxe3o-dos-eixos-na-mesa-projetada.-autoria-pruxf3pria.}

Suas dimensões físicas foram definidas para proporcionar a integração
com o robô, seus motores foram dimensionados para ter torque e
velocidade suficiente para rotacionar tubos ou peças com até 8kg de
massa.

O \textbf{Eixo 1} é onde será realizada a rotação da peça acoplada para
ser soldada, neste eixo é necessário realizar o controle de velocidade e
posição e não necessita de um torque tão significante como no outro
eixo, dessa forma a utilização de um motor CC é a melhor alternativa, já
que é possível variar sua velocidade e sentido de rotação, assim como,
implementar um controle em cascata com eventuais perturbações, também é
independente de ângulo de passo e pode fazer um travamento da peça em
praticamente qualquer ângulo entre 0° a 360°, aumentando a precisão da
máquina. Outro fator importante considerado é o custo, sendo o motor CC
muito mais barato para a realização desta aplicação.

O motor escolhido para o \textbf{Eixo 1} é um motor de corrente contínua
com motor-redutor, torque de 30kgf.cm e 90 RPM, com corrente máxima de
4A e tensão DC 12V, com uma potência de 10W.

Esse motor é posicionado diretamente no eixo de rotação, sendo
devidamente isolado e fixado na estrutura, paralelo a ele irá um sensor
do tipo encoder incremental responsável por fazer a captação e leitura
da velocidade e posição do mesmo, enviando esses valores a um Arduino,
que será o controlador de toda a mesa.

Para o \textbf{Eixo 2,} foi necessário a realização de uma redução
mecânica para que não seja necessário utilizar um motor com um torque
muito alto, o que em termos de custo, se torna inviável. Como ainda é
necessário um torque entre 30 a 40 kgf.cm o motor CC poderia ser
utilizado, porém como a variável de controle mais significativa nesse
eixo é o posicionamento e não mais a velocidade, um motor de passo pode
ser utilizado.

O motor então escolhido foi um Motor de Passo Nema 34, com ângulo de
passo de 1.8°, podendo ser melhorado através do Driver de Passo, com um
torque de Holding\footnote{Uma propriedade específica do
  \href{https://www.sciencedirect.com/topics/engineering/stepper-motor}{motor
  de passo} é o torque de retenção ou holding torque . Isso significa
  que uma corrente contínua é extraída da fonte de alimentação para
  garantir que o eixo do motor de passo seja mantido estacionário e não
  gire quando submetido a forças externas. {[}2{]}} de 52kgf.cm e
corrente/fase de 5A.

Esse motor será posicionado na parte interior da máquina logo abaixo da
engrenagem maior da redução mecânica. Seus parâmetros de controle são
capturados diretamente pelo Driver de acionamento do motor e repassados
ao Arduino.

\includegraphics[width=6.52604in,height=3.77943in]{media/image14.png}

\includegraphics[width=5.23966in,height=3.55729in]{media/image27.png}

\paragraph{Figura 2. Vista Posterior da mesa (Autoria
própria).}\label{figura-2.-vista-posterior-da-mesa-autoria-pruxf3pria.}

\textbf{Ajustes Necessários}

O projeto passou por alterações de layout e correções para que tivesse
êxito no atendimento dos requisitos necessários para sequência do
projeto integrador, substituição das laterais estruturais para correção
de problemas como empenamento, desalinhamento e erros de corte de
material que ocorreram durante o processo de montagem da estrutura e dos
itens do equipamento.

A solda da estrutura passou pelo processo de remoção das soldas
excedentes e refeito solda da base bem como o restante das soldas
estruturais refeitas e concluídas, a estrutura da base do eixo de
posição passou por alterações de layout com remoção parcial do eixo 02
de movimentação e remoção da base fixa de acoplamento do motor de cc do
eixo 01 a fim de contemplar o novo formato do motor e sensor que foram
implementados no projeto. Foi alterado o layout do sistema de posição do
eixo 02 para que a estrutura se adequar ao motor de passo como
alongamento do eixo do motor e nova furação de fixação. O eixo de
posição e movimentação 01 passou novamente pelo processo de usinagem
para correção do excesso do diâmetro do material, o acoplamento do eixo
01 passou por ajustes para que se adequar a correia de movimentação do
encoder.

O projeto demanda de alinhamento preciso dos dois eixos de movimentação
e posição para que não tenha colisões estruturais bem como o
funcionamento correto dos eixos a centralização foi realizada com
auxílio de equipamentos de precisão para alinhamento e balanceamento
estrutural para que tenha uma compensação dos erros de construção
mecânica, para ajustes do nivelamento da mesa foram instalados ``pés''
ajustáveis que melhoram a estabilidade significativamente compensando os
desníveis do local de instalação.

\subsection{Ajustes mecanicos}\label{ajustes-mecanicos}

Neste primeiro momento não foram encontrados erros em decorrência de
desalinhamento e desbalanceamento estrutural que provocam perturbações
significativas ou que apresentam risco de colisão entre estrutura e base
de posicionamento, foram utilizados pesos para realização do teste com
carga dentro do limite estabelecido no projeto, os teste de controle do
controlador em malha fechada proporcionaram resultados dentro do
esperado com respostas que permitem observar a não linearidade do
controlador onde é possível observar a zona morta de acionamento do
motor e a zona de saturação do controlador

\textbf{Fluxograma de funcionamento elétrico}

\includegraphics[width=6.18750in,height=3.09386in]{media/image35.png}

Figura N. Fluxograma elétrico. Autoria Própria.

\subsection{2.1 Projeto elétrico}\label{projeto-eluxe9trico}

Para o projeto elétrico foi utilizado o software (CircuitMaker) de
esquemático de ligações elétricas.

\includegraphics[width=6.26772in,height=4.63889in]{media/image23.png}

\paragraph{Figura 3. Esquemático do circuito elétrico. (Autoria
própria).}\label{figura-3.-esquemuxe1tico-do-circuito-eluxe9trico.-autoria-pruxf3pria.}

\textbf{Tabela de tags da parte de controle e instrumentação}

\begin{longtable}[]{@{}ll@{}}
\toprule
INPUT DESCRIÇÃO & OUTPUT DESCRIÇÃO\tabularnewline
\midrule
\endhead
\textbf{D2} Microcontrolador & \textbf{Fase B} encoder\tabularnewline
\textbf{D3} Microcontrolador & \textbf{Fase A} encoder\tabularnewline
\textbf{D5} Microcontrolador & \textbf{IN1} ponte H\tabularnewline
\textbf{D6} Microcontrolador & \textbf{IN4} ponte H\tabularnewline
\textbf{D9} Microcontrolador & \textbf{IN2} ponte H\tabularnewline
\textbf{D10} Microcontrolador & \textbf{IN3} ponte H\tabularnewline
\textbf{D11} Microcontrolador & \textbf{DIR+} Step Driver\tabularnewline
\textbf{D12} Microcontrolador & \textbf{ENA+} Step Driver\tabularnewline
\textbf{D13} Microcontrolador & \textbf{PUL+} Step Driver\tabularnewline
\textbf{GND} Microcontrolador & \textbf{GND} Encoder

\textbf{PUL- / DIR- / ENA-} StepDriver

\textbf{GND} ponte H\tabularnewline
\textbf{5VCC} Microcontrolador & \textbf{Vss / ENA / ENB} ponte H

\textbf{V} encoder\tabularnewline
\bottomrule
\end{longtable}

\textbf{Tabela de tags acionamento dos motores}

\begin{longtable}[]{@{}ll@{}}
\toprule
INPUT DESCRIÇÃO & OUTPUT DESCRIÇÃO\tabularnewline
\midrule
\endhead
\textbf{ISENA} Microcontrolador & \textbf{GND}
microcontrolador\tabularnewline
\textbf{ISENB} Microcontrolador & \textbf{GND}
microcontrolador\tabularnewline
\textbf{V+} motor CC & \textbf{OUT2 / OUT3} ponte H\tabularnewline
\textbf{V-} motor CC & \textbf{OUT1 / OUT4} ponte H\tabularnewline
\textbf{12Vcc} Fonte & \textbf{Vs} ponte H\tabularnewline
\textbf{D10} Microcontrolador & \textbf{IN3} ponte H\tabularnewline
\textbf{B-} Step Driver & \textbf{B2} Step Driver\tabularnewline
\textbf{B+} Step Driver & \textbf{B1} Step Driver\tabularnewline
\textbf{A-} Step Driver & \textbf{A2} Step Driver\tabularnewline
\textbf{A+} Step Driver & \textbf{A1} Step Driver\tabularnewline
\textbf{V+} Step Driver & \textbf{48VCA} Saída Tensão
trafo\tabularnewline
\textbf{GND} Step Driver & \textbf{GND} trafo\tabularnewline
\bottomrule
\end{longtable}

\textbf{Tabela de tags restantes}

\begin{longtable}[]{@{}ll@{}}
\toprule
INPUT DESCRIÇÃO & OUTPUT DESCRIÇÃO\tabularnewline
\midrule
\endhead
\textbf{220VCA} & \textbf{NC1} emergência

\textbf{L} trafo

\textbf{ON} \textbf{/ OFF}

\textbf{COM} sensor fim de curso\tabularnewline
\textbf{ON} \textbf{/ OFF} & \textbf{L} fonte 12VCC\tabularnewline
\textbf{NEUTRO} & \textbf{NEUTRO} fonte 12VCC

\textbf{N} trafo\tabularnewline
\bottomrule
\end{longtable}

\section{Controle e
Instrumentação}\label{controle-e-instrumentauxe7uxe3o}

Os seguintes itens correspondem à segunda etapa da realização do
projeto, equivalente à disciplina de ``Projeto Integrador 2''. Nessa
etapa serão idealizados e implementados conceitos de Controle e
Instrumentação.

\subsection{Requisitos do cliente}\label{requisitos-do-cliente}

Visando atender a aplicação dos conhecimentos adquiridos a respeito de
Controle e Instrumentação foi proposto o atendimento a três requisitos
do cliente nessa etapa, que são o controle de velocidade, controle de
posição ambos no eixo 1, controle de posição em malha aberta do eixo 2.

\subsection{Detalhamento funcional}\label{detalhamento-funcional}

As funções parciais e elementares do projeto, na área de controle e
instrumentação estão demonstradas conforme o diagrama abaixo.

\includegraphics[width=6.28646in,height=2.79517in]{media/image31.png}

\paragraph{Figura 4. Funções de controle e instrumentação. (Autoria
própria).}\label{figura-4.-funuxe7uxf5es-de-controle-e-instrumentauxe7uxe3o.-autoria-pruxf3pria.}

\subsection{Matriz morfológica de controle e
instrumentação}\label{matriz-morfoluxf3gica-de-controle-e-instrumentauxe7uxe3o}

A matriz morfológica traz as possíveis soluções para realizar as funções
propostas no detalhamento funcional. O objetivo da matriz é mostrar para
cada uma das funções elementares descritas, as possíveis soluções
disponíveis dentro do contexto do projeto.

Dentre as possibilidades apresentadas em cada um dos tópicos, os
destacados em verde foram utilizados.

\begin{longtable}[]{@{}llll@{}}
\toprule
\textbf{Sensores de velocidade e posição} & Encoder incremental

\includegraphics[width=1.10532in,height=0.86269in]{media/image22.jpg} &
Encoder Absoluto

\includegraphics[width=0.94602in,height=0.94602in]{media/image32.png} &
Resolver

\includegraphics[width=1.08383in,height=1.09351in]{media/image39.png}\tabularnewline
\midrule
\endhead
\textbf{Sensores de posição limite - Fim de curso} & Eletromecânico

\includegraphics[width=1.13021in,height=1.09295in]{media/image7.png} &
Eletromagnético

\includegraphics[width=0.90981in,height=0.90981in]{media/image21.png} &
Indutivo

\includegraphics[width=1.01252in,height=1.00356in]{media/image26.png}\tabularnewline
\textbf{Acionamento Motor CC} & L298N

\includegraphics[width=1.16701in,height=1.16701in]{media/image11.png} &
TIP 120

\includegraphics[width=1.25035in,height=1.25035in]{media/image37.png} &
L293

\includegraphics[width=1.16504in,height=1.15660in]{media/image25.png}\tabularnewline
\textbf{Acionamento motor de passo} & Stepper-Driver
AC\includegraphics[width=1.18463in,height=0.89064in]{media/image33.png}
& &\tabularnewline
\textbf{Controle dos motores } & Arduino UNO

\includegraphics[width=1.03394in,height=1.03394in]{media/image19.png} &
Arduino Mega

\includegraphics[width=1.13292in,height=0.83602in]{media/image38.png}
&\tabularnewline
\bottomrule
\end{longtable}

\paragraph{Tabela 1 - Componentes de controle e instrumentação.(Autoria
própria).}\label{tabela-1---componentes-de-controle-e-instrumentauxe7uxe3o.autoria-pruxf3pria.}

\subsection{Matriz de avaliação de controle
instrumentação}\label{matriz-de-avaliauxe7uxe3o-de-controle-instrumentauxe7uxe3o}

Para auxiliar na tomada de decisão a matriz de avaliação consiste em
manipular os dados disponíveis a partir da matriz morfológica, cada
tabela corresponde aos itens elencados anteriormente na tabela 1.

Selecionados três itens de avaliação para cada matriz é possível
distribuir pesos de importância (0 a 10) e em seguida, realizar a
avaliação para cada item incluindo notas para os mesmos. Com a tabela e
valores montados pode-se calcular a média de cada item e por fim
escolher o mais adequado.

\begin{longtable}[]{@{}lllll@{}}
\toprule
\textbf{Sensores de velocidade e posição} & \textbf{Peso} &
\textbf{Encoder incremental} & \textbf{Encoder absoluto} &
\textbf{Resolver}\tabularnewline
\midrule
\endhead
Custo & \textbf{10} & 9 & 3 & 1\tabularnewline
Resolução & \textbf{10} & 9 & 7 & 10\tabularnewline
Precisão & \textbf{9} & 7 & 10 & 7\tabularnewline
\textbf{Total} & & \textbf{8.1} & \textbf{6.3} &
\textbf{5.7}\tabularnewline
\bottomrule
\end{longtable}

\begin{longtable}[]{@{}lllll@{}}
\toprule
\textbf{Sensores de posição limite - Fim de curso} & \textbf{Peso} &
\textbf{Mecânico} & \textbf{Magnético} &
\textbf{Indutivo}\tabularnewline
\midrule
\endhead
Custo & \textbf{10} & 9 & 10 & 6\tabularnewline
Sensibilidade & \textbf{8} & 6 & 6 & 8\tabularnewline
Praticidade & \textbf{9} & 10 & 8 & 5\tabularnewline
\textbf{Total} & & \textbf{7.6} & \textbf{7.3} &
\textbf{5.6}\tabularnewline
\bottomrule
\end{longtable}

\begin{longtable}[]{@{}lllll@{}}
\toprule
\textbf{Acionamento Motor CC} & \textbf{Peso} & \textbf{L298N} &
\textbf{TIP120} & \textbf{L293}\tabularnewline
\midrule
\endhead
Custo & \textbf{10} & 10 & 10 & 5\tabularnewline
Corrente de pico & \textbf{10} & 9 & 0 & 10\tabularnewline
Tensão de Operação & \textbf{9} & 9 & 10 & 10\tabularnewline
\textbf{Total} & & \textbf{9} & \textbf{6.3} & \textbf{8}\tabularnewline
\bottomrule
\end{longtable}

\begin{longtable}[]{@{}lllll@{}}
\toprule
\textbf{Acionamento do motor de passo} & \textbf{Peso} &
\textbf{Driver-stepper AC} & &\tabularnewline
\midrule
\endhead
Custo & \textbf{10} & 10 & &\tabularnewline
Tensão de operação & \textbf{7} & 7 & &\tabularnewline
Confiabilidade & \textbf{9} & 10 & &\tabularnewline
\textbf{Total} & & \textbf{8} & &\tabularnewline
\bottomrule
\end{longtable}

\begin{longtable}[]{@{}lllll@{}}
\toprule
\textbf{Controle dos motores} & \textbf{Peso} & \textbf{Arduino UNO} &
\textbf{Arduino MEGA} &\tabularnewline
\midrule
\endhead
Custo & \textbf{10} & 10 & 6 &\tabularnewline
Tensão de operação & \textbf{7} & 7 & 7 &\tabularnewline
Confiabilidade & \textbf{9} & 10 & 10 &\tabularnewline
\textbf{Total} & & \textbf{8} & \textbf{6.6} &\tabularnewline
\bottomrule
\end{longtable}

\paragraph{Tabela 2 - Matriz avaliação de controle e instrumentação.
(Autoria
própria).}\label{tabela-2---matriz-avaliauxe7uxe3o-de-controle-e-instrumentauxe7uxe3o.-autoria-pruxf3pria.}

\begin{enumerate}
\def\labelenumi{\arabic{enumi}.}
\item ~
  \subsection{Métodos práticos}\label{muxe9todos-pruxe1ticos}

  \begin{enumerate}
  \def\labelenumii{\arabic{enumii}.}
  \item ~
    \subsubsection{Definição dos controladores para os
    eixos}\label{definiuxe7uxe3o-dos-controladores-para-os-eixos}
  \end{enumerate}
\end{enumerate}

Devido a necessidade de controlar a velocidade no eixo 1, e posição em
ambos os eixos, e necessitar de um motor de passo no eixo 2, foi optado
por um controlador em malha aberta no eixo 2, e um controlador em
cascata de velocidade e posição no eixo 1.

\subsubsection{Hardware necessário para o sistema de controle e
instrumentação}\label{hardware-necessuxe1rio-para-o-sistema-de-controle-e-instrumentauxe7uxe3o}

Por meio das conclusões obtidas a partir da matriz de avaliação em
associação com a matriz morfológica foi possível determinar o Hardware
que seria utilizado no sistema de controle e instrumentação, sendo este
descrito logo abaixo:

\begin{itemize}
\item
  \begin{quote}
  1 Microcontrolador Arduíno Uno
  \end{quote}
\item
  \begin{quote}
  1 Driver-stepper AC
  \end{quote}
\item
  \begin{quote}
  1 Ponte H L298N
  \end{quote}
\item
  \begin{quote}
  1 Encoder Incremental (E6B2-CWZ6C)
  \end{quote}
\end{itemize}

Na sequência, estes componentes serão apresentados.

\begin{itemize}
\item
  \begin{quote}
  \textbf{Microcontrolador Arduino Uno}
  \end{quote}
\end{itemize}

O microcontrolador é utilizado para as variáveis controladas do sistema,
combinado com o sistema de controle elaborado por meio de análises
realizadas pelo grupo é possível alcançar os objetivos de controle
desejados, sendo eles velocidade e posicionamento.

\begin{quote}
Os dados retirados do datasheet do Arduino são:
\end{quote}

\begin{itemize}
\item
  \begin{quote}
  Memória: AVR CPU até 16 MHz; 32KB Flash 2KB SRAM; 1KB EEPROM
  \end{quote}
\item
  \begin{quote}
  Periféricos: 2x 8-bit Timer/Counter com um período dedicado do
  registrador e canais comparadores; 1x 16-bit Timer/Counter com um
  período dedicado de registrador, input de captura e canal comparador;
  1x USART com baud rate gerador fraccionado; 1x controlador serial
  periférico de interface (SPI); 1x Dual mode controlador I2C; 1x
  Comparador analógico com uma input de referência escalada; Watchdog
  Timer com um on-chip oscillator separado; Seis canais PWM Interrupt e
  wake-up em mudança de pino.
  \end{quote}
\item
  \begin{quote}
  Alimentação: 2.7- 5.5 volts.
  \end{quote}
\item
  \begin{quote}
  Os seguintes pinos estarão em uso no projeto: GND; 5V; D11; D10; D9;
  D6; D5; D4; D3; D2. Na página 6, no projeto elétrico é possível
  verificar as conexões.
  \end{quote}
\item
  \begin{quote}
  Uma placa de circuito foi elaborada de forma manual para melhor
  conexão, evitando mal contato nos pinos.
  \end{quote}
\end{itemize}

\begin{itemize}
\item
  \begin{quote}
  \textbf{Ponte H L298N}
  \end{quote}
\end{itemize}

A utilização da Ponte H faz-se necessária para poder inverter o sentido
de rotação do motor CC

\begin{quote}
Os dados da Ponte H retirados do datasheet da mesma são:
\end{quote}

\begin{itemize}
\item
  \begin{quote}
  Tensão de alimentação operacional até 46 V; Corrente total DC até 4 A;
  Baixa tensão de saturação; Proteção de sobretemperatura; Tensão de
  entrada Lógica de 0 até 1.5 V com imunidade a altos ruídos.
  \end{quote}
\end{itemize}

\begin{itemize}
\item
  \begin{quote}
  \textbf{Encoder incremental E6B2-CWZ6C}
  \end{quote}
\end{itemize}

Para que seja possível fazer o controle da planta é necessário alguns
sensores, sendo o principal para o projeto o encoder, o qual fará a
leitura de velocidade e posição do sistema de rotação.

\begin{itemize}
\item
  \begin{quote}
  Os dados do encoder encontram-se abaixo e foram retirados do datasheet
  oficial do sensor:
  \end{quote}
\end{itemize}

Tensão de alimentação 5 VDC a 24 VDC +15\%, ripple: 5\% máx; Corrente de
consumo 80 mA máx; Resolução (pulsos/rotação) 10, 20, 30, 40, 50, 60,
100, 200, 300, 360, 400, 500, 600, 720, 800, 1,000, 1,024, 1,200, 1,500,
1,800, 2,000; Fases de saída A, B e Z; Configuração de saída coletor
aberto NPN; Resposta máxima de frequência 100 kHz; Tempos de subida e
descida da saída 1 µs máx; Velocidade máxima permitida 6000 r/min;
Temperatura de operação de -10 a 70°C; Peso aproximado 100 g; Zona morta
(momento de inércia) 1×10−6 kg·m2 max.; 3 × 10−7 kg·m2 max. at 600 P/R
max.

\begin{itemize}
\item
  \begin{quote}
  \textbf{Driver-stepper AC}
  \end{quote}
\end{itemize}

O Drive do motor de passo servirá para acionamento, proteção e controle
do motor de passo implementado no projeto. O modelo utilizado em questão
possui diversas correntes e Pulsos/revolução, sendo selecionados via
switch de seleção, de acordo com a necessidade

\begin{itemize}
\item
  \begin{quote}
  Tensão de alimentação 18VAC a 80 VAC; Corrente de referência 2A,
  2.57A, 3.14A, 3.71A, 4.28A, 4.86A, 5.43A, 6A; Corrente de pico 2.4A,
  3.08A, 3.77A, 4.45A, 5.14A, 5.83A, 6.52A, 7.20A; Pulsos/revolução 400,
  800, 1600, 3200, 6400, 12800, 25600, 51200, 1000, 2000, 4000, 5000,
  10000, 20000, 40000.
  \end{quote}
\end{itemize}

\begin{enumerate}
\def\labelenumi{\arabic{enumi}.}
\item ~
  \section{Métodos e análises}\label{muxe9todos-e-anuxe1lises}

  \begin{enumerate}
  \def\labelenumii{\arabic{enumii}.}
  \item ~
    \subsection{Controle de posição em malha
    aberta}\label{controle-de-posiuxe7uxe3o-em-malha-aberta}

    \begin{enumerate}
    \def\labelenumiii{\arabic{enumiii}.}
    \item ~
      \subsubsection{Coleta de dados}\label{coleta-de-dados}
    \end{enumerate}
  \end{enumerate}
\end{enumerate}

Devido ao controle de posição do eixo 2 ser de malha aberta, não há
coleta de dados em tempo real por sensores, então será analisado a
partir da entrada de pulsos uma relação entre a movimentação do eixo e o
número de pulsos por segundo enviados para a lógica do driver-stepper.

\subsubsection{Definição do
controlador}\label{definiuxe7uxe3o-do-controlador}

Como um controle de malha aberta consiste em aplicar um sinal de
controle, esperando-se que ao final de um tempo a variável controlada
atinja um determinado valor. Nesse sistema de controle não serão
utilizadas informações obtidas a partir da coleta contínua de dados.
Mais especificamente, o sinal de controle não é calculado a partir de
uma medição do sinal de saída, conforme demonstrado na figura abaixo :

\includegraphics[width=4.94792in,height=1.43750in]{media/image5.png}

\begin{quote}
\textbf{Figura 5. Diagrama de controle em malha aberta}
\end{quote}

\subsubsection{Implementação}\label{implementauxe7uxe3o}

Posteriormente a análise do sistema em malha aberta, uma relação entre a
quantidade de passos, é dimensionada, para isso alguns parâmetros são
considerados, como a relação de 10:1 entre as engrenagens a equação do
controlador será implementada na plataforma do Arduino por meio de
código conforme será descrito no Apêndice A.

\begin{enumerate}
\def\labelenumi{\arabic{enumi}.}
\item ~
  \subsection{Controle de Velocidade e posição em
  cascata}\label{controle-de-velocidade-e-posiuxe7uxe3o-em-cascata}

  \begin{enumerate}
  \def\labelenumii{\arabic{enumii}.}
  \item ~
    \subsubsection{Coleta de dados}\label{coleta-de-dados-1}
  \end{enumerate}
\end{enumerate}

Serão realizadas coletas de dados para identificação da função de
transferência do sistema variando o fornecimento de potência via
alteração da razão cíclica do PWM. Após serem definidas as faixas de
coletas, os dados que forem coletados serão analisados via software
MATLAB, utilizando de uma toolbox chamada \emph{ident}, onde a será
possível gerar uma função de transferência G(s), a qual possuirá uma
taxa de aproximação, denominada Best Fits, que determina a taxa de
aproximação da função obtida com os dados coletados.

A partir da função de transferência G(s) será possível analisar a
resposta a um degrau unitário da planta em malha aberta e também
observar um tempo de assentamento (Ts), utilizando então essas
informações será definido o controlador, conforme a se explicará
brevemente a seguir.

\subsubsection{Definição do
controlador}\label{definiuxe7uxe3o-do-controlador-1}

Após a análise dos dados, será utilizado um sistema de controle em
cascata, porque as variáveis de controle são diretamente
correlacionadas, de modo que considerando uma função de posição como
\textbf{θn} é possível derivá-la para obter uma função de velocidade
\textbf{θ'n}, e da mesma forma é possível integrar \textbf{θ'n}, obtendo
\textbf{θn}, por causa dessa relação é possível utilizar o método de
cascata, para simplificar o controle do sistema.

Graças a esse método que é amplamente utilizado no controle de braços
robóticos, haverá no sistema um controlador para velocidade, que ficará
numa malha secundária, seguido por um integrador, tornando o controlador
de velocidade dependente do controlador de posição, conforme diagrama a
seguir:\includegraphics[width=6.26772in,height=1.63889in]{media/image6.png}

\paragraph{Figura 6. Diagrama de controle em cascata de um braço
robótico.}\label{figura-6.-diagrama-de-controle-em-cascata-de-um-brauxe7o-robuxf3tico.}

\subsubsection{Implementação}\label{implementauxe7uxe3o-1}

Após a análise do sistema, a equação do controlador em cascata será
implementada na plataforma do Arduino por meio de código.

Com o posicionamento do motor e do encoder incremental utilizado no
projeto, e a partir de um código desenvolvido foi possível coletar
dados, com variáveis de controle definidas pelo grupo. Estes dados foram
analisados e tratados, dos mesmos foi possível retirar dois gráficos
informativos sobre o comportamento do motor CC de acordo com o valor do
PWM escolhido.

Utilizando uma tensão de entrada constante de 14,5 V e selecionando
valores do PWM a partir de 5 até 255, incrementado de 5 em 5 é possível
medir a tensão de saída real para o motor e comparando com uma tensão
estimada calculada (figura 6) a partir de \(Ve = \frac{PWM*14,5}{255}\)

\includegraphics[width=5.11979in,height=3.09569in]{media/image17.png}

\textbf{Figura 6. Gráfico (Tensão real - tensão estimada) x PWM (Autoria
Própria)}

É possível perceber uma zona morta no motor, onde ele não é ativado,
pois a tensão de saída do mesmo não é suficiente para acioná-lo. Essa
zona morta está nos valores de PWM de 5 até 40, fazendo com que o motor
comece a rotacionar somente com o valor de PWM=45.

No gráfico abaixo (tensão de saída x PWM) é possível observar essa zona
morta e o instante em que o mesmo começa seu movimento (Tensão saída=
2V) de forma constante de acordo com valores de PWM.

\includegraphics[width=6.26772in,height=2.23611in]{media/image20.png}

\textbf{Figura 7. Gráfico Tensão saída x PWM (Autoria Própria)}

Considerando os dados anteriores é possível afirmar que há uma perda de
tensão na ponte H, que pode ser representada a partir da equação
\(y = (0,05017*x) - 0,1607\), obtida pela ferramenta \emph{Curve
Fitting} do software Matlab.

Tendo isso em consideração na seguinte coleta de dados foi utilizado um
filtro projetado para amenizar esses efeitos a fim de gerar uma planta
mais fiel para o modelo.

Na seguinte coleta de dados, a fim de obter uma planta que atuasse na
região desejada de operação e desconsiderasse a região de zona morta do
motor e a região de não linearidade que é até 30 RPM, os dados foram
coletados em um intervalo de 102 a 255 PWM, aplicando diversos sinais do
tipo degrau de diferentes amplitudes, coletando um bom número de
amostras por degrau e utilizando um período de amostragem de 0,02
segundos foram coletados 215 dados, conforme a tabela exemplo abaixo:

\begin{longtable}[]{@{}ll@{}}
\toprule
u & y\tabularnewline
\midrule
\endhead
204.00 & 14.75\tabularnewline
204.00 & 43.75\tabularnewline
204.00 & 58.25\tabularnewline
204.00 & 65.00\tabularnewline
204.00 & 67.50\tabularnewline
204.00 & 69.25\tabularnewline
204.00 & 70.00\tabularnewline
204.00 & 70.50\tabularnewline
204.00 & 70.75\tabularnewline
\bottomrule
\end{longtable}

\paragraph{Tabela 4. Demonstração dos dados coletados.(Autoria própria)
}\label{tabela-4.-demonstrauxe7uxe3o-dos-dados-coletados.autoria-pruxf3pria}

Onde u é a entrada do degrau em PWM, e y é a saída do encoder já em RPM,
então os dados foram tratados no Matlab utilizando o \emph{framework
ident} para gerar uma função de transferência para o modelo. Conforme
abaixo:

\(G(s) = \frac{2,447s + 76,31}{s^{2} + 40,73s + 225,4}\)

\subparagraph{}\label{section}

\subparagraph{Equação (1)}\label{equauxe7uxe3o-1}

Agora com a planta do sistema em mãos é possível analisar o
comportamento do sistema. Segundo, Nise(verificar ano e referência) é
possível separar em três tipos as respostas de um sistema de segunda
ordem com uma entrada ao degrau e as respectivas influências de seus
pólos em suas respostas que são:

O sistema subamortecido, sobreamortecido e criticamente amortecido.

\begin{itemize}
\item
  \begin{quote}
  O sistema subamortecido possui dois pólos (raízes da equação
  característica) complexos e conjugados;
  \end{quote}
\item
  \begin{quote}
  O sistema sobreamortecido possui dois pólos reais e distintos;
  \end{quote}
\item
  \begin{quote}
  O sistema criticamente amortecido possui dois pólos reais iguais;
  \end{quote}
\end{itemize}

Analisando graficamente obtemos:

\begin{itemize}
\item
  \begin{quote}
  Sistema subamortecido:
  \end{quote}
\end{itemize}

\(T(s) = \frac{9}{s^{2} + 2s + 9}\)

\begin{quote}
\includegraphics[width=6.26772in,height=3.26389in]{media/image12.png}
\end{quote}

\paragraph{ Figura x. Resposta ao degrau de um sistema subamortecido.
Fonte(Autoria
própria)}\label{figura-x.-resposta-ao-degrau-de-um-sistema-subamortecido.-fonteautoria-pruxf3pria}

\includegraphics[width=6.26772in,height=3.38889in]{media/image28.png}

\paragraph{Figura x. Lugar das raízes de um sistema subamortecido.
Fonte(Autoria
própria)}\label{figura-x.-lugar-das-rauxedzes-de-um-sistema-subamortecido.-fonteautoria-pruxf3pria}

\begin{itemize}
\item
  \begin{quote}
  Sistema sobreamortecido:
  \end{quote}
\end{itemize}

\begin{quote}
\(T(s) = \frac{9}{s^{2} + 9s + 9}\)
\end{quote}

\paragraph{\texorpdfstring{\protect\includegraphics[width=6.26772in,height=3.25000in]{media/image29.png}\textbf{Figura
x. Resposta ao degrau de um sistema s}obre\textbf{amortecido.
Fonte(Autoria
própria)}}{Figura x. Resposta ao degrau de um sistema sobreamortecido. Fonte(Autoria própria)}}\label{figura-x.-resposta-ao-degrau-de-um-sistema-sobreamortecido.-fonteautoria-pruxf3pria}

\paragraph{\texorpdfstring{\protect\includegraphics[width=6.27083in,height=3.32039in]{media/image10.png}\textbf{Figura
x. Lugar das raízes de um sistema s}obrea\textbf{mortecido.
Fonte(Autoria
própria)}}{Figura x. Lugar das raízes de um sistema sobreamortecido. Fonte(Autoria própria)}}\label{figura-x.-lugar-das-rauxedzes-de-um-sistema-sobreamortecido.-fonteautoria-pruxf3pria}

\begin{itemize}
\item
  \begin{quote}
  Sistema criticamente amortecido:
  \end{quote}
\end{itemize}

\(T(s) = \frac{9}{s^{2} + 6s + 9}\)

\includegraphics[width=6.26772in,height=3.33333in]{media/image4.png}

\paragraph{Figura x. Resposta ao degrau de um sistema criticamente
amortecido. Fonte(Autoria
própria)}\label{figura-x.-resposta-ao-degrau-de-um-sistema-criticamente-amortecido.-fonteautoria-pruxf3pria}

\paragraph{\texorpdfstring{\protect\includegraphics[width=6.26772in,height=3.36111in]{media/image24.png}\textbf{Figura
x. Lugar das raízes de um sistema} criticamente \textbf{amortecido.
Fonte(Autoria
própria)}}{Figura x. Lugar das raízes de um sistema criticamente amortecido. Fonte(Autoria própria)}}\label{figura-x.-lugar-das-rauxedzes-de-um-sistema-criticamente-amortecido.-fonteautoria-pruxf3pria}

Tendo em vista esses conceitos de comportamento do sistema então a
planta é analisada a fim de definir o comportamento da mesma, a seguir é
possível ver a resposta ao degrau e o lugar das raízes da
planta.\includegraphics[width=6.26772in,height=3.38889in]{media/image2.png}

\paragraph{Figura x. Resposta ao degrau da planta do
sistema.}\label{figura-x.-resposta-ao-degrau-da-planta-do-sistema.}

\paragraph{\texorpdfstring{\protect\includegraphics[width=6.26772in,height=3.37500in]{media/image1.png}\textbf{Figura
x.} Lugar das raízes \textbf{da planta do
sistema.}}{Figura x. Lugar das raízes da planta do sistema.}}\label{figura-x.-lugar-das-rauxedzes-da-planta-do-sistema.}

Como é possível ver nas figuras acima a planta dois pólos reais e
distintos, -6,61 e -34,1 e um zero em -31,2, logo o sistema é
sobreamortecido.

Então foram iniciados os projetos dos controladores em cascata levando
isso em consideração. Conforme previamente abordado para o controle de
velocidade o sistema será controlado por um controle PID em cascata,
onde foi definido o modelo do PID como:

\(C(s) = Kp + Ki*\frac{1}{s} + Kd*s\)

Tendo em vista que os requisitos para controle do sistema eram de:

\begin{itemize}
\item
  \begin{quote}
  Erro em Regime permanente aproximadamente 0;
  \end{quote}
\item
  \begin{quote}
  Overshoot menor que 10\%;
  \end{quote}
\item
  \begin{quote}
  Tempo de assentamento aproximadamente 1 segundo;
  \end{quote}
\end{itemize}

Após a definição dos requisitos de controle, é dado início ao projeto do
controle PID de velocidade, utilizando a equação anterior de PID que
pode ser reescrita da seguinte forma:

\includegraphics[width=3.93750in,height=0.85417in]{media/image18.png}

A partir da equação acima é mais claro que para o projeto do controlador
será necessário alocar um polo em zero e dois zeros (\(z_{1},\ z_{2}\)),
para isso então foi utilizado a ferramenta \emph{rltool} do Matlab, onde
os requisitos foram adicionados na opção de design da ferramenta a fim
de gerar um guia para o projeto.

\includegraphics[width=6.26772in,height=2.94444in]{media/image15.png}

\paragraph{Figura x. Planta do sistema e os designs no
rltool.}\label{figura-x.-planta-do-sistema-e-os-designs-no-rltool.}

Para definir os dois zeros do controlador foi utilizado

Tendo esses requisitos em mente foi calculado o controlador PID de
velocidade, com os seguintes parâmetros.

\begin{quote}
\textbf{Kp} = 0.53282;

\textbf{Ki} = 11.9894;

\textbf{Kd} = 0.059198;

Gerando assim o seguinte controlador:

\(C(s) = \frac{0,0592*(s^{2} + 9,001s + 202,5)}{s}\)
\end{quote}

\subparagraph{Equação(2)}\label{equauxe7uxe3o2}

A resposta do sistema com o controlador a um degrau pode ser vista na
figura(10):\includegraphics[width=6.26772in,height=3.23611in]{media/image3.png}

\paragraph{Figura 10. Resposta ao degrau do sistema
controlado.}\label{figura-10.-resposta-ao-degrau-do-sistema-controlado.}

Onde é possível ver que todos requisitos foram atendidos, assegurando
assim um bom controle de velocidade para o sistema.

Como é necessário além do controlador de velocidade outro para
posicionamento é utilizado o método em cascata, onde os controladores
são colocados em sequência atuando em conjunto no sistema.

Para o controle de posição o projeto inicial previa outro PID, porém
conforme o projeto avançou foi visto que os termos Ki e Kd tendiam a
zero em todas propostas, devido a robustez do controlador de velocidade,
então o controlador acabou resumindo-se a um Controlador P, onde é
calculado apenas o Kp, a partir de agora chamado de Kp2.

Como o modelo da planta está em RPM, é necessário uma conversão para o
modelo de posição visto que a integração de RPM, não gera um constante
que o sensor possa fazer a leitura, devido a leitura em quadratura do
encoder e a relação de redução entre o eixo do motor CC e o encoder,
cada volta do motor corresponde a 12000 passos do encoder.

Considerando uma velocidade de 40 RPM como a velocidade máxima para uma
solda consistente, logo o controlador de posição não pode mandar um
sinal maior que 40 para o controlador de velocidade, então a partir
disso é possível fazer dizer que para um sinal de referência de 1 volta,
o controlador de posição deve receber um sinal de 12000, enquanto o
controlador de posição deve receber um sinal de 40.

Tendo esses parâmetros em mente, foi calculado um Kp2 utilizando uma
regra de três, resultando em aproximadamente 0,03333 da seguinte forma:

Kp2 está para 40, como 12000 está para 1.

\(\frac{Kp*12000}{40*1} = \ 1\ \  \rightarrow \ Kp\  = \ \frac{40}{12000} = 0.0033\)

Levando todas informações anteriores em consideração foi necessário uma
atualização do diagrama de blocos, conforme na figura a seguir:

\paragraph{\texorpdfstring{\protect\includegraphics[width=6.27083in,height=1.34925in]{media/image16.png}Figura
11. Diagrama de blocos da malha de controle
atualizado.}{Figura 11. Diagrama de blocos da malha de controle atualizado.}}\label{figura-11.-diagrama-de-blocos-da-malha-de-controle-atualizado.}

\subsubsection{Discretização}\label{discretizauxe7uxe3o}

Para a discretização do controlador de velocidade foi adotado uma
estratégia de discretização onde é considerado a equação do PID já no
domínio do tempo, e também a discretização é feita termo a termo, a
partir dos métodos integradores digitais e derivadores digitais são
obtidas as equações a diferenças.

Para a discretização do controlador foi utilizado o método backward de
Euler. Desta forma, a função do controlador discreta C(z), pode ser
obtida a partir de C(s), através da seguinte equação:

\subparagraph{\texorpdfstring{\protect\includegraphics[width=1.59531in,height=0.68052in]{media/image36.png}
}{ }}\label{section-1}

\subparagraph{Equação (3)}\label{equauxe7uxe3o-3}

Com isso a função transferência do controlador, descrita na equação (2)
é substituindo a equação (3) gerando o seguinte resultado:

\subparagraph{\texorpdfstring{\protect\includegraphics[width=3.41061in,height=0.46219in]{media/image13.png}
}{ }}\label{section-2}

\subparagraph{Equação(4)}\label{equauxe7uxe3o4}

\textbf{Resultados práticos}

\begin{quote}
\textbf{Conclusao}
\end{quote}

\section{Referências
bibliográficas}\label{referuxeancias-bibliogruxe1ficas}

Robótica ABB, Acesso em 18 de out. 2021. Disponível em:
\textless{}\href{https://new.abb.com/products/robotics/pt}{\emph{https://new.abb.com/products/robotics/pt}}\textgreater{};

{[}2{]}Denise Ying , ... Hongliang Ren , em
\href{https://www.sciencedirect.com/book/9780128175958/flexible-robotics-in-medicine}{Robótica
Flexível em Medicina} , 2020, disponível em: \textless{}
\href{https://www.sciencedirect.com/topics/engineering/holding-torque}{\emph{https://www.sciencedirect.com/topics/engineering/holding-torque}}\textgreater{},
Acesso em: 05 de mar. 2022

Livro: soldagem fundamentos e tecnologia ,3 edição, editora UFMG de
2009, AUTORES: Paulo Villani Marques, Paulo José Modenesi, Alexandre
Queiroz Bracarense;

Acesso em 22 de out. 2021 Disponível em:
\textless{}\href{https://www.filipeflop.com/categoria/motores/}{\emph{https://www.filipeflop.com/categoria/motores/}}\textgreater{};

Schneider Eletric, Acesso em 22 de ago 2021. Disponível em:
\textless{}\href{https://www.regulacni-pohony.cz/soubor/manual-bmh-2016-03-en-pdf}{\emph{https://www.regulacni-pohony.cz/soubor/manual-bmh-2016-03-en-pdf}}\textgreater{};

De Motor, Acesso em 22 de out. 2021. Disponível em:
\textless{}\href{https://pt.demotor.net/motores-electricos/motores-de-corrente-alternada}{\emph{https://pt.demotor.net/motores-electricos/motores-de-corrente-alternada}}\textgreater{};

CNC, Acesso em 23 de out. 2021. Disponível em:
\textless{}\href{https://www.amazon.com/Router-Machine-Rotational-2-Phase-Stepper/dp/B07G94C6DX}{\emph{https://www.amazon.com/Router-Machine-Rotational-2-Phase-Stepper/dp/B07G94C6DX}}\textgreater{};

Eixo Rotativo, Acesso em 23 de out. 2021. Disponível em:
\textless{}\href{https://pt.aliexpress.com/item/32881756224.html}{\emph{https://pt.aliexpress.com/item/32881756224.html}}\textgreater{};

Tubos de aço, Acesso em 30 de out. 2021. Disponível em:
\textless{}\href{https://acotubo.com.br/wp-content/uploads/2016/08/ACO_005_Catalogos_Acotubo2016_OnLine_02_TubosAco.pdf}{\emph{https://acotubo.com.br/wp-content/uploads/2016/08/ACO\_005\_Catalogos\_Acotubo2016\_OnLine\_02\_TubosAco.pdf}}\textgreater{}

Datasheet Arduino UNO, Acesso em 01 de fev. 2022. Disponível em:
\textless{}\href{https://docs.arduino.cc/resources/datasheets/A000066-datasheet.pdf}{\emph{https://docs.arduino.cc/resources/datasheets/A000066-datasheet.pdf}}\textgreater{}

Siciliano et al, Robotics -- Modelling planning and control, 2009

\section{Apêndice A - Códigos
Arduino}\label{apuxeandice-a---cuxf3digos-arduino}

\subsection{Bibliotecas utilizadas}\label{bibliotecas-utilizadas}

No projeto foram utilizadas duas bibliotecas a fim de otimizar a
performance do Arduino, Encoder.h e PID\_v1.h, ambas disponíveis na
interface de programação disponibilizada pelo fabricante.

\subsection{Por que utilizar uma
biblioteca?}\label{por-que-utilizar-uma-biblioteca}

Uma biblioteca se assemelha a uma função criada pelo usuário, a
diferença é que é que há mais agilidade, pois pode-se usar programas
distintos; É possível escrever códigos menores e mais organizados.
Assim, se reduz a chance de erros e retrabalhos; Ela facilita a
atualização de programas diversos. Como previamente dito foram
utilizadas as bibliotecas Encoder.h e PID\_v1.h, descritas a seguir:

\begin{itemize}
\item ~
  \subsection{Encoder.h}\label{encoder.h}
\end{itemize}

Essa biblioteca é a responsável pela leitura dos pulsos do encoder,
torna-se necessário o uso da mesma devido a necessidade de performance
na leitura, já que uma leitura otimizada requer uma pesquisa muito
maior.

\begin{itemize}
\item ~
  \subsection{PID\_v1.h}\label{pid_v1.h}
\end{itemize}

Essa biblioteca é responsável pelo cálculo do PID, utilização dessa
justifica-se por dois motivos, reduzir o tamanho do código principal, e
otimização do código, porque a biblioteca trabalha com um tempo de
amostragem real e um alocamento de memória otimizado, gerando um tempo
de resposta menor até a ação do controlador. Um comparativo entre os
dois códigos é demonstrado a seguir:

\includegraphics[width=4.68750in,height=3.18750in]{media/image8.png}

\paragraph{Figura A1. Parte do código implementado na
biblioteca.}\label{figura-a1.-parte-do-cuxf3digo-implementado-na-biblioteca.}

\includegraphics[width=5.23958in,height=3.17708in]{media/image9.png}

\paragraph{Figura A2. Parte do Código implementado pelo
Grupo.}\label{figura-a2.-parte-do-cuxf3digo-implementado-pelo-grupo.}

Como pode ser observado pelas imagens anteriores, a parte principal de
ambos códigos é praticamente idêntica, ocorrendo apenas uma diferença
que onde, foi utilizado um tempo de amostragem T, como um valor fixo na
figura A2, o mesmo é um valor dinâmico na biblioteca, onde o mesmo acaba
gerando mais precisão na hora do controlador atuar sobre o sistema.

\subsection{Código Utilizado
Completo}\label{cuxf3digo-utilizado-completo}

\#include \textless{}PID\_v1.h\textgreater{}

\#include \textless{}Encoder.h\textgreater{}

//declaracao dos pinos utilizados para controlar o motor

const int PINO\_IN1 = 5;

const int PINO\_IN4 = 6;

const int PINO\_IN2 = 9;

const int PINO\_IN3 = 10;

//Variaveis do encoder

Encoder Enc(2, 3);

float EncAtual = 0.0;

float EncAnterior = 0.0;

double T = 0.02;//Período de amostragem (T = )

unsigned long lastSend = 0;

//Define variaveis pro pid de velocidade

double Setpoint, vSetpoint, Input, Output;

//Variaveis para o controle de posicao

double pSetpoint, ErroP, Cp;

double P0 = 0.0; //Posicao inicial

float DEnc = 0.0; //Diferenca entre leituras

float VEnc = 0.0; // Velocidade em Rpm

//Parametros dos controladores

double Kp = 0.53282;

double Ki = 11.9894;

double Kd = 0.059198;

Kp2= 0.003183098862

PID myPID(\&Input, \&Output, \&Setpoint, Kp, Ki, Kd, DIRECT);

//Timer 1 interrupcao

ISR(TIMER1\_COMPA\_vect) \{

EncAtual = (Enc.read()); //recebe o valor da interrupção

DEnc = EncAtual - EncAnterior;

EncAnterior = EncAtual;

VEnc = (60 * DEnc / (12000 * T));

\}

//Configuracao timer1

void Tempo\_Config\_1(void)

\{

TCCR1B = 0;

TCCR1A = 0;

TIMSK1 = 0;

// Timer de 16bits, conta até 65535

TCCR1B \textbar{}= (1 \textless{}\textless{} WGM12) \textbar{} (1
\textless{}\textless{} CS12); //Timer em CTC e prescaler de 256

//TCCR1B \textbar{}= (1 \textless{}\textless{} WGM12) \textbar{} (1
\textless{}\textless{} CS10)\textbar{} (1 \textless{}\textless{} CS12);
//Timer em CTC e prescaler de 1024

TIMSK1 = (1 \textless{}\textless{} OCIE1A); //Interrupção no COMPA

//Com prescaler de 8, no modo CTC f = fcpu/(Prescaler*(1+OCR1A))

//f = 16Hz

//T = 0.02s

OCR1A = ((16000000 / 256) * T) - 1;

\}

void setup() \{

Tempo\_Config\_1(); //Seta o tempo de interrupção

Serial.begin(115200);

//configuracao dos pinos como saida

pinMode(PINO\_IN1, OUTPUT);

pinMode(PINO\_IN2, OUTPUT);

pinMode(PINO\_IN3, OUTPUT);

pinMode(PINO\_IN4, OUTPUT);

//inicia o codigo com os motores parados

digitalWrite(PINO\_IN1, LOW);

digitalWrite(PINO\_IN2, LOW);

digitalWrite(PINO\_IN3, LOW);

digitalWrite(PINO\_IN4, LOW);

//Inicia led de controle

pinMode(LED\_BUILTIN, OUTPUT);

P0 = Enc.read();

Input = 0;

Setpoint = 0;

myPID.SetMode(AUTOMATIC);

\}

void loop() \{

pSetpoint = 12000; //Setpoint para 1 volta

vSetpoint = 40;

ErroP = (pSetpoint + P0) - (EncAtual);

Input = EncAtual * Kp2;

Cp = min(((P0 + pSetpoint) * Kp2), vSetpoint);

Setpoint = Cp;

myPID.Compute();

//Define sentido

if (pSetpoint \textgreater{} 0) \{

analogWrite(PINO\_IN1, Output);

analogWrite(PINO\_IN4, Output);

\} else \{

Cp = min(((P0 - pSetpoint) * Kp2), vSetpoint);

Setpoint = Cp;

myPID.Compute();

analogWrite(PINO\_IN2, Output);

analogWrite(PINO\_IN3, Output);

\}

if (millis() - lastSend \textgreater{} 80) \{

lastSend = millis();

Serial.print(Setpoint);

Serial.print(" ");

Serial.print(ErroP);

Serial.print(" ");

Serial.print(Input);

//Serial.print(" ");

// Serial.print(Output);

Serial.println(" ");

\}

\}

\end{document}
